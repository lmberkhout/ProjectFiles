\documentclass{article}
\usepackage{float,amsmath}
\usepackage{graphicx}
\usepackage{color}
\usepackage[letterpaper,margin=1in]{geometry}
\usepackage{hyperref}

%\setlength{\textwidth}{6.5in}

\begin{document}

\author{HERA Team}
\title{Roadmap for HERA Network Configuration and Bandwidth}
\maketitle

\section{Introduction}
HERA is an international experiment to detect and characterize the Epoch of
Reionization (EOR).  The telescope is located at the South African SKA site in
the Karoo Astronomy Reserve.  This note summarizes the overall network
configuration and bandwidth for HERA as relevant to the interfaces with SKA-SA
infrastructure.  HERA construction and observing are proceeding in parallel.

The HERA correlator is currently located next to the telescope array. In late 2017 or early 2018, the correlator will be moved to the KAPB. At that time the correlator will ingest raw voltage streams at terabit rates. These will be fed over a bundle of $\sim$96 fibers from the array to the KAPB. This future upgrade is outside the scope of this summary, other than to note the future bandwidth requirements to the US.

\section{HERA network requirements prior to new correlator in 2018}
Figure~\ref{fig:hi_level} shows the very high-level network configuration for
2017 and 2018.  The correlator currently lives in the `HERA container' in the
middle of the HERA telescope array.  The correlator data files are transferred for processing and storage in the CMC container using a dedicated fiber.  The processed files are then transferred off-continent to the USA, currently to a cluster at the University of Pennsylvania in Philadelphia, PA.  Over the next month or two, the CMC-based equipment will be moved to the KAPB and the USA node will be moved to the National Radio Astronomy Observatory (NRAO) site in Socorro, NM.  The single pair of 1Gbe fiber between the HERA container and the processing cluster is adequate for operations through fall of 2017.

\begin{figure}[H]
\includegraphics[width=0.8\textwidth]{network.pdf}
\centering
\caption{High-level organization of network.  See text for explanation.}
\label{fig:hi_level}
\end{figure}

Prior to the move of the equipment from the CMC to the KAPB, there is a desire
to update the network configuration to accommodate the evolving needs of the
KAT/SKA network and HERA.  There is no strict requirement that this must
happen before the physical move, but doing so allows the network change to be
deployed separately from the major hardware relocation.  When the equipment is relocated to the KAPB additional storage will be added to the system and the switches will be upgraded.


Figure~\ref{fig:net_org} shows a schematic of the proposed network arrangement. In order to ensure smooth HERA operations through mid-2018, we provide the following \textbf{requirements}:

\begin{enumerate}
\item There must be a login portal allowing secure access to HERA nodes from the Internet.
\item There must be 32 IP addresses reserved for HERA nodes on the SKA/KAT network so that they may be accessed directly from our login portal.
\item Some HERA nodes must be able to access to the SKA/KAT network for integration with CAM (as both data sinks and data sources).
\item There must be a dedicated fiber pair between the HERA container and the HERA processing cluster (currently located in the CMC; soon KAPB).
\item The network must support a data rate to the US of 200~Mbps by mid-2017, growing to 400~Mbps by mid-2018.
\end{enumerate}

\begin{figure}[H]
\includegraphics[width=\textwidth]{HERA_2017_network_organization.pdf}
\centering
\caption{Schematic of proposed network arrangement.}
\label{fig:net_org}
\end{figure}

\section{Data Rate}
HERA currently records at a data rate of 200~Mbps. In 2018 this will increase to 1.5~Gbps. Currently the average bandwidth to the US is 60~Mbps, so we can only transfer a small fraction of our data. This limits analysis capabilities and means that mission-critical data are not replicated off-site. Moving data at the currently-attainable rate of 60~Mbps causes conflict with SKA~SA staff Internet use, so copies to the US are limited to night-time hours, further reducing the effective bandwidth available to the project.

The desired data rate to support current operations is 200~Mbps. The target for mid-2018 is 400~Mbps.

\section{Action Plan}
Options are limited by only having a single fiber into the KAPB available for HERA use and by a desire to have a single connection point for all HERA machines onto the SKA network. This will limit the possibility of loopbacks.  On the HERA side there is a desire to let several machines be able to access the outside world and some smaller subset of machines to be visible on the SKA net.  In the current setup, HERA is connected to the legacy KAT network, which is kept wholly separate from the SKA net, and because most machines are behind a NAT it is difficult to monitor traffic and bandwidth usage from individual machines. The KAT network is being phased out.  

In the new setup HERA will be connected to a single level 3 port on the KAPB Cisco network and given a subnet of some range of IPs. Using virtual networks in the HERA switches a private network and a public network will be created at the ska connection point and then both networks will be extended out to the HERA container. See diagrams in Figure \ref{fig:network_drawings}.  The change will proceed in two steps. First we will make all the necessary networking changes, tapping into the KAPB via the available fiber and creating the networks in the CMC. Second we will move the HERA processing machines and the main connection point into the KAPB, replacing the CMC switch with a fiber pass through.

\begin{figure}[H]
\includegraphics[width=0.49\textwidth]{HERA_network_Jan2017.jpg}
\includegraphics[width=0.49\textwidth]{HERA_network_2017_CMC.jpg}
\includegraphics[width=0.49\textwidth]{HERA_network_2017_KAPB.jpg}
\caption{Notes from meeting Feb 13 2017. Clockwise from top left, the current networking setup with the HERA container and the HERA processing cluster in the CMC, the proposed intermediate setup moving HERA off the KAT network and onto the KAPB Cisco network and moving from two fibers (private and KAT) to a single fiber hosting two virtual nets, then bottom left moving the processing cluster into the KAPB, replacing the router in the CMC with a fiber pass-through. \label{fig:network_drawings}}
\end{figure}

\end{document}
